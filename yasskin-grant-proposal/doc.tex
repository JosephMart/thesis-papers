\documentclass[a4paper,11pt]{article}
\usepackage[T1]{fontenc}
\usepackage[utf8]{inputenc}
\usepackage{lmodern}

\title{Textbook Problem Dependency Web}
\author{Joseph Martinsen}

\begin{document}
\maketitle

This thesis looks to tackle a problem that arises after a textbook
has been written, and published. To customize the book for a particular
audience, it may be desirable to reorder some of the chapters.
However, there may be dependencies among the chapters, examples and exercises
which make it very tedious to rearrange the order of not only the chapters but
also the associated problems.

Given a textbook with chapters \(A, B, C\) where the material in \(B\) and \(C\)
depend on \(A\) but not each other. Suppose the original order is  \(A, B, C\),
and some of \(C\)'s problems utilize of information from chapter\(B\) as well as \(A\).
If we wish to reorder the chapters as \(A, C, B\) then the new order
is acceptable in terms of content dependencies but some of the problems from \(C\)
need to be moved to \(B\).

I believe this process of reorganizing chapters, sections, and problems can all
be automatized utilizing a dependency tree built and managed within a graph
database.

\end{document}