\documentclass[a4paper,11pt]{article}
\usepackage[T1]{fontenc}
\usepackage[utf8]{inputenc}
\usepackage{lmodern}

\title{Textbook Problem Dependency Web}
\author{Joseph Martinsen}

\begin{document}
\maketitle

This thesis looks to tackle the problem that arises when a textbook
has been carefully and meticulous designed, written, and published.
Once the textbook is "done", it is very tedious to change around
the order of not only the chapters but the associated problems.

Given a textbook with chapters $A, B, C$ in the order. Let $C$'s topic depend on 
chapter $A$ content and $C$'s problems comprise of information from chapters
$A$ and $B$. If the order were to be changed to $A, C, B$, the new order
is acceptable in terms of content dependencies but the problems will need to be
reordered.

I believe this process of reorganization chapters, sections, and problems can all
be automatized utilizing a dependency tree built and managed within a graph
database.

\end{document}
