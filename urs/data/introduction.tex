%%%%%%%%%%%%%%%%%%%%%%%%%%%%%%%%%%%%%%%%%%%%%%%%%%%
%
%  New template code for TAMU Theses and Dissertations starting Fall 2016.
%
%
%  Original Author: Sean Zachary Roberson
%  This version adapted for URS by Parasol lab.
%  Adapted from version 3.16.10, which was last updated on 9/29/2016.
%  URS adaptation last updated 1/9/2017.
%
%%%%%%%%%%%%%%%%%%%%%%%%%%%%%%%%%%%%%%%%%%%%%%%%%%%
%%%%%%%%%%%%%%%%%%%%%%%%%%%%%%%%%%%%%%%%%%%%%%%%%%%%%%%%%%%%%%%%%%%%%%
%%                           SECTION I
%%%%%%%%%%%%%%%%%%%%%%%%%%%%%%%%%%%%%%%%%%%%%%%%%%%%%%%%%%%%%%%%%%%%%


\pagestyle{plain} % No headers, just page numbers
%\pagenumbering{arabic} % Arabic numerals
%\setcounter{page}{1}

\chapter{INTRODUCTION AND LITERATURE REVIEW}

\section{Background Info}

Textbooks go through a long and arduous process before a student or professor is able to view and use it. This process requires much work and effort into not only validating the content of the textbook but also validating the ordering of the textbook as whole. Much like a jigsaw puzzle, each chapter fits one after another based on the dependency of the topics being taught. On top of these chapters being ordered, the exercises must also be placed in the correct place in order to not give an exercise that is based on a topic that has not been presented to the reader previously.

After all this work on ordering has been completed (among other meticulous things), the textbook may finally ready for publication. As with many good textbooks, many professors and institutions may enjoy the content within the textbook but would prefer delivering the content to a student in a different order than what the current order the textbook is in currently. Most of the time, the work and effort that has gone into meticulously ordering the chapters and exercises must now be revisited again and modified. This process is nearly as time consuming and laborious as the first going through this process.

\section{Existing Technology}

In the current field, most re-orderings of chapters and exercises are done by hand with human intervention by either textbook authors or commercial institutions. An exception is TopHat (TM) who, for their interactive textbooks, has the ability to reorder in a drag and drop manner but one lacking feature is any sort of feedback of any sort being provided to the user. I was able to test drive this product and it felt much like a changing the order of a power point presentation as it is in its current state. Besides the technology features that TopHat offers on the rest of their platform, which is outside the scope of this discussion, much of what their reordering seems to achieve can simply be done by reordering the pages of the textbook in some sort of PDF viewer.

From speaking with several Math professors at the Joint Math Meeting 2019, this idea of being able to reorder structural components of their textbook is an issue they commonly face and is not brand new. From each professor I interacted with, there were not aware of a simply and useful that would help them complete this task. There was one platform for identifying dependencies between subjects and textbooks that then allows a professor to generate a curriculum based off these topics, but this is done on a much higher textbook level, not by chapter, section or exercise.

\section{Approach}

My approach to this situation wants to take on a similar approach to what TopHat has done but builds heavily on the fact that the original author has an idea of the flow of a textbook, including changes to the order the repercussions that may result. While it is not feasible for a professor to provide this feedback in person, the realtime feedback aspect can be preserved by creating a platform that \textit{understands} the original authors concerns and worries and then provide them to the consumer of the textbook the moment a change they make may result in possible conflict of dependencies.
