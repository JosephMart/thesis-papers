%%%%%%%%%%%%%%%%%%%%%%%%%%%%%%%%%%%%%%%%%%%%%%%%%%%
%
%  New template code for TAMU Theses and Dissertations starting Fall 2016.
%
%
%  Original Author: Sean Zachary Roberson
%  This version adapted for URS by Parasol lab.
%  Adapted from version 3.16.10, which was last updated on 9/29/2016.
%  URS adaptation last updated 1/9/2017.
%
%%%%%%%%%%%%%%%%%%%%%%%%%%%%%%%%%%%%%%%%%%%%%%%%%%%
%%%%%%%%%%%%%%%%%%%%%%%%%%%%%%%%%%%%%%%%%%%%%%%%%%%%%%%%%%%%%%%%%%%%%%
%%                           SECTION I
%%%%%%%%%%%%%%%%%%%%%%%%%%%%%%%%%%%%%%%%%%%%%%%%%%%%%%%%%%%%%%%%%%%%%


\pagestyle{plain} % No headers, just page numbers
%\pagenumbering{arabic} % Arabic numerals
%\setcounter{page}{1}

\chapter{INTRODUCTION AND LITERATURE REVIEW}

\section{Background Info}

Textbooks go through a long and arduous process before a student or professor is able to view and use it. This process requires much work and effort into not only validating the content of the textbook but also validating the ordering of the textbook as whole. Much like a jigsaw puzzle, each chapter fits one after another based on the dependency of the topics being taught. On top of these chapters being ordered, the exercises must also be placed in the correct place in order to not give an exercise that is based on a topic that has not been presented to the reader previously.

After all this work on ordering has been completed (among other things), the textbook is ready for publication. For many good textbooks, many professors and institutions may enjoy the content within the book but would prefer delivering the content to a student in a different order than is in the textbook currently. Most of the time, work and effort that has gone into meticulously ordering the chapters and exercises must be now be revisited again and modified. This process is nearly as time consuming and laborious as the first time.

\section{Existing Technology}

From what I have been able to find, most re-orderings of chapters and exercises are done by hand with human intervention. Pearson (TM) has the ability to reorder their own published textbooks but it is not applicable for the general textbook.

From speaking with several Math professors at the Joint Math Meeting 2019, this idea is not brand new but there is no platform for them to do this task. There is a platform for identifying dependencies between subjects and textbooks that then allows a professor to generate a curriculum based off these topics, but this is done on a textbook level, not by chapter or exercise.
