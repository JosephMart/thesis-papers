%%%%%%%%%%%%%%%%%%%%%%%%%%%%%%%%%%%%%%%%%%%%%%%%%%%
%
%  New template code for TAMU Theses and Dissertations starting Fall 2016.
%
%
%  Original Author: Sean Zachary Roberson
%  This version adapted for URS by Parasol lab.
%  Adapted from version 3.16.10, which was last updated on 9/29/2016.
%  URS adaptation last updated 1/9/2017.
%
%%%%%%%%%%%%%%%%%%%%%%%%%%%%%%%%%%%%%%%%%%%%%%%%%%%
%%%%%%%%%%%%%%%%%%%%%%%%%%%%%%%%%%%%%%%%%%%%%%%%%%%%%%%%%%%%%%%%%%%%%%%
%%%                           SECTION II
%%%%%%%%%%%%%%%%%%%%%%%%%%%%%%%%%%%%%%%%%%%%%%%%%%%%%%%%%%%%%%%%%%%%%%

\chapter{THE ALGORITHM}

\section{The Motivation for the Algorithm}

\textit{traverse old and new textbooks
diff algorithm between two trees}

\textit{traverse old and new textbooks
diff algorithm between two trees}

\cite{bile}
\cite{tsur}
\cite{reactReconcile}

\section{The Design}

tree of unit (chapters/sections/pages) order -- 1a orig and 1b modified

tree of exercise order -- 2a orig and 2b modified

tree of unit interdependences 3

tree of exercises dependencies on units 4

flexiblility 

1. author creates original trees 1a, 2a, 3, 4

2. adopting institution or instructor specifies desired order of units 1b \newline
   program provides warnings of impermissible orders as this tree is created \newline
   adopting institution or instructor can override order
   
3. program automatically creates 2b \newline
   adopting institution or instructor previews exercise and can override order
   
4. actually reorder the files of the text and links between them

% \begin{figure}[ht]
% \centering
% \includegraphics[scale=0.75]{TAMUthesis_CMD_windows.png}
% \caption[The command line compiler in Windows.]{The command line compiler in Windows. It is not suggested that you compile using this method. See compilation instructions in the README.}

% \label{fig:CMD_1}
% \end{figure}


% \begin{table}[h!]
% 	\centering
% 	\label{Band}
% 	\caption{Scores from the 2011 Arcadia Festival of Bands.}
%         \vspace{1em}
% 	\begin{tabular}{|l|l|l|}
% 		\hline
% 		School Name & Band Score & Auxiliary Score \\ \hline
% 		Rancho Bernardo & 96.15 & 89.15 \\ \hline
% 		Mt. Carmel & 95.30 & 83.55 \\ \hline
% 		Riverside King & 93.85 & 91.75 \\ \hline
% 		Diamond Bar & 93.20 & 88.60 \\ \hline
% 		El Dorado & 92.80 & 95.45 \\ \hline
% 		Chino & 92.65 & 91.45 \\ \hline
% 		Henry J. Kaiser & 92.60 & 87.55 \\ \hline
% 		Glendora & 92.60 & 89.15 \\ \hline
% 		Montebello & 90.50 & 82.70 \\ \hline
% 		Mira Mesa & 89.65 & 91.50 \\ \hline
% 	\end{tabular}
% \end{table}

% %Make other examples.
% \begin{equation} \label{Equ.2.1}
% y=c_1\cos(t)+c_2\sin(t)
% \end{equation}
% \begin{equation} \label{Equ.2.2}
% e^{it}=\cos(t)+i\sin(t)
% \end{equation}

% Equation \ref{Equ.2.1} is the general solution to the differential equation $y''+y=0$. In the source code, the \textit{ref} command allows you to refer to an equation by a label you created. References must be made after the equation has been created; attempting to refer to an equation before it is defined results in a question mark placeholder. Some more sample equations are below. Notice the first set below is not numbered.

% \begin{align*}
% \log (x^n) &= \log (x \cdot x \cdot \ldots \cdot x) \\
% &= \log x + \log x + \ldots + \log x \\
% &= n \log x
% \end{align*}
% \begin{equation} \label{Equ.2.3}
% X^T X \mathbf{u} = X^T \mathbf{y}
% \end{equation}
% \begin{equation}\label{Equ.2.4}
% u(x, t) = \int_{-\infty}^{\infty} G(x, \tau) \exp\left(-\frac{(t-\tau)^2}{4kt}\right) \ d\tau
% \end{equation}
% \begin{gather}
% \mathcal{L}(f) = \int_{0}^{\infty} e^{-st} f(t) \ dt \\
% \begin{split} \label{Equ.2.5}
% \mathcal{F}(f) = \frac{1}{2\pi}\int_{-\infty}^{\infty} e^{i \omega x} f(x) \ dx
% \end{split}
% \end{gather}

% You can use labels to refer to equations you create. \ref{Equ.2.5} is the \textbf{Laplace transform} used extensively in differential equations. \ref{Equ.2.3} is the matrix representation of the \textbf{normal equations} used in least-squares regression.
