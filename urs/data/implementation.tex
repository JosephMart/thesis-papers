%%%%%%%%%%%%%%%%%%%%%%%%%%%%%%%%%%%%%%%%%%%%%%%%%%%
%
%  New template code for TAMU Theses and Dissertations starting Fall 2016.
%
%
%  Original Author: Sean Zachary Roberson
%  This version adapted for URS by Parasol lab.
%  Adapted from version 3.16.10, which was last updated on 9/29/2016.
%  URS adaptation last updated 1/9/2017.
%
%%%%%%%%%%%%%%%%%%%%%%%%%%%%%%%%%%%%%%%%%%%%%%%%%%%
%%%%%%%%%%%%%%%%%%%%%%%%%%%%%%%%%%%%%%%%%%%%%%%%%%%%%%%%%%%%%%%%%%%%%%
%%                           IMPLEMENTATION
%%%%%%%%%%%%%%%%%%%%%%%%%%%%%%%%%%%%%%%%%%%%%%%%%%%%%%%%%%%%%%%%%%%%%

\chapter{THE IMPLEMENTATION}

\section{The Textbook}

For the implementation of the design, I have gone with a proof of concept model. By applying this workflow to a portion of a single textbook for the purpose of working out any workflow issues and testing the algorithm in a real world situation. The textbook that will first be integrated is MYMath Apps Calculus by Dr. Yasskin and Dr. Meade. The prototype used the first two parts of Calculus 2 on Techniques and Application of Integration. During the time frame of this project, I was only able to implement nets 1, 2 and 4.

\section{Textbook Build Process}

The textbook is web based and as such, is developed to conform to web standards. Each page is written in Hypertext Markup Language (HTML). In order to better clarify as well as lighten the load on the author(s) of the textbook, a JavaScript object notation (JSON) file defines the order of the textbook. This file is then used by an template engine called Nunjucks. Nunjucks allows for portions of each HTML page that are not a part of the content of a unit to be specified in a template. This includes headers and footers of each page. Each actual unit of HTML is then written, one HTML page per unit. By using the JSON file, Nunjucks is then able to combine them and produce the entire textbook. The order of the JSON file is similar to net 4a as previously presented. From this ordering, the textbook is then built and structured. When net 4a is modified to net 4b, the JSON file is rewritten and the book may be recompiled. 


\section{The Tech Stack}

\subsection{Database}

A graph database is utilized for storing the different nets. The particular database used was Neo4j. This database allowed for a DAG to be stored in it's entirety. It also allows for the finding of a particular node by simply querying for information about the node. There is also an graphical interface provided by this service that visually shows related nodes and the related DAGs for a node found by the query as shown in Fig \ref{fig:graph}.

\subsection{Server}

The purpose of the server was to fetch and store data to the database and perform the computational portions of the algorithm. This included identifying warning messages and ensuring that all dependencies are met for each change or modification that a editor makes. 

\subsection{Client}

The client utilized ReactJS. This is a front end library that has existing modules that allows for easily modifying and reordering a JSON tree. This was not a simple implementation so it did require a good amount of tweaking in order for it to fit the necessary use case.

\section{Integration with Textbook}

Full integration with the textbook involves two different parts. One is adding all the textbook nets for use by the platform and the other is being able to export the necessary data in order to actually modify the textbook. The prototype  used had net 4a completed. With some effort and additional mapping, nets 1 and 2 were also constructed.

\subsection{Import Process}

The import process requires the original author to upload all five nets. From there the server is able to validate and identify if there are any possible dependency violations. If there are, the author is then provided messages as to what dependencies are not met and allows the author to modify them using the graphical user interface. This graphical user interface is the same as the one the editor uses for reordering later on. The prototype had no violations.

\subsection{Reorder Process}

This is the process where the editor has the control to modify the structure of the units using the interface shown in Fig \ref{fig:nestedChapters}. After each modification or tweak, the new and updated tree is sent to the server for verification. The server then verifies dependencies and identifies if any have been violated. Depending on the order of the violation and where in the tree the violation occurred, a message is generated to inform the editor of what the violations are and possible suggestions as to how to fix the ordering.

After the editor has modified the ordering of the units, they are able to review these changes. This review screen highlights changes, additions or deletions of units. This screen allows the user to see all the changes they are attempting to do before they commit them. Once the user approves these changes, this new ordering is then stored as the new ordering for the particular editor.

\subsection{Export Process}

Due to the structure of the build process of the prototype in use, the export process is rather trivial. It is simply an export of net 4a into the JSON format. This allows for the existing build process to generate the new ordering of the textbook as specified by the editor.
