%%%%%%%%%%%%%%%%%%%%%%%%%%%%%%%%%%%%%%%%%%%%%%%%%%%
%
%  New template code for TAMU Theses and Dissertations starting Fall 2016.
%
%
%  Original Author: Sean Zachary Roberson
%  This version adapted for URS by Parasol lab.
%  Adapted from version 3.16.10, which was last updated on 9/29/2016.
%  URS adaptation last updated 1/9/2017.
%
%%%%%%%%%%%%%%%%%%%%%%%%%%%%%%%%%%%%%%%%%%%%%%%%%%%
%%%%%%%%%%%%%%%%%%%%%%%%%%%%%%%%%%%%%%%%%%%%%%%%%%%%%%%%%%%%%%%%%%%%%%
%%                           CONCLUSION
%%%%%%%%%%%%%%%%%%%%%%%%%%%%%%%%%%%%%%%%%%%%%%%%%%%%%%%%%%%%%%%%%%%%%

\chapter{CONCLUSION}

\section{Challenges}

One challenge encountered was the fact that the textbook was already written without this design or thoughts about the algorithm. In order to get the textbook into a place where it could be integrated with the algorithm required some slight modifications and additions. In the algorithm's current state, this is what most textbooks are required to do in order to fully integrate into the platform. This textbook used did have a similar tree structure as tree 1 (the ordering of units). With some modifications of this tree, it became the same format as required.

Another large problem encountered is related to the compounding complexity of the project. Each chapter, section and page was treated as simply as a unit. While for the purposes of dependency mapping, this group was allowable, for the use of the author of the textbook, each entity is different in both information they contain and how they may be treated for the purpose of writing the textbook.

\textit{time to work on exercise}

\section{Broader Impact}

The main focus of the thesis thus far has been with a single textbook, but a broader goal is to allow any textbook to utilize this platform with the underlying algorithm. Allowing an author to use this, especially during the initial design of the textbook, will allow seamless reordering to the textbook in the event that an adopting institution or professor decides to modify the original ordering of the textbook.

\section{Future Plans}

The integration portion of the platform and textbook has not been entirely completed. There has been tweaking and modifications to the platform, the build ordering of the textbook and easily converting them into to use by the platform. A goal is to have complete integration with the textbook to the point that it can actually be tied into the build process. As a build step, any modifications can be done using the platform and then immediately carried out by producing a new version of the textbook with the appropriate modifications.

\textit{time to work on exercise}
